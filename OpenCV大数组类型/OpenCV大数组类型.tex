\chapter{OpenCV大数组类型}
在OpenCV中,cv::Mat类被用来表示任何维度的密度矩阵,%
其中cv::Mat针对的是密集连续型的存储,%
大多数的图型数据都是被保存为这种类型,%
即使数据为空,%
预留的存储空间仍然存在。

\section{Mat内存布局}
对于一维的矩阵,其元素可以被认为是连续存储的。%
对于二维的矩阵,其数据是按行进行存储的,%
对于三维的矩阵,其中每个平面是是按行进行排列,%
然后每个平面进行排列。

Mat类包含一个公有的\textbf{\color{magenta}flags}数据成员标识数据的内容,%
\textbf{\color{magenta}dims}来标识矩阵的维度,%
\textbf{\color{magenta}rows}和\textbf{\color{magenta}cols}
标识矩阵的行数和列数(对于dims>2的矩阵,这两个数据成员不可用),%
\textbf{\color{magenta}data}用来表示数据存储的首地址,%
\textbf{\color{magenta}refcount}表示其指针\textbf{\color{magenta}cv::Ptr<>}引用的次数,%
其中cv::Ptr<>的行为类似于C++中的智能指针。%
对于data中的数据布局可以通过矩阵的\textbf{\color{magenta}step[~]}进行描述,%
对于索引为$(i_{0},i_{1},\dots,i_{N_{d}-1})$的元素,%a
其地址可以表示为:
\begin{equation}
  \begin{split}
    \&\left(mtx_{i_{0},i_{1},\dots,i_{N_{d}-1},N_{d}}\right) & = mtx.data + mtx.step[0]*i_{0}
    + mtx.step[1]*i_{1} \\
    & + \cdots + mtx.step[N_{d}-1]*i_{N_{d}-1}
  \end{split}
\end{equation}
对于二维的简单情形可以表示为:
\begin{equation}
  \&(mtx_{i,j}) = mtx.data + mtx.step[0]*i + mtx.step[1]*j
\end{equation}

\section{Mat数据访问方式}
对于Mat中的数据,最直接的访问方式是使用模板函数\textbf{\color{magenta}at$<>$()}。%
例如:
\begin{cppcode}
  cv::Mat m = cv::Mat::eye(10, 10, 32FC1);
  std::cout << "Element (3,3) is "
            << m.at<float>(3,3) << std::endl;
\end{cppcode}

对于多通道矩阵,对于每个像素(element)的访问形式最简单就是使用cv::Vec$<>$,如:
\begin{cppcode}
  cv::Mat m = cv::Mat::eye(10, 10, 32FC2);
  printf(
    "Element (3,3) is (\%f,\%f)\n",
    m.at<cv::Vec2f>(3,3)[0],
    m.at<cv::Vec2f>(3,3)[1]
    );
\end{cppcode}
          
完整的例子如下:
\begin{cppcode}
  #include <iostream>
  #include "opencv2/opencv.hpp"

  int main(int argc, char *argv[]){
    cv::Mat grayimg(600, 800, CV_8UC1);
    cv::Mat colorimg(600, 800, CV_8UC3);

    for (int i = 0; i != grayimg.rows; ++i){
      for(int j = 0; j != grayimg.cols; ++j){
        grayimg.at<uchar>(i, j) = (i+j)%255;
      }
    }

    for (int i = 0; i != colorimg.rows; ++i){
      for (int j = 0; j != colorimg.cols; ++j){
        cv::Vec3b pixel;
        pixel[0] = i % 255;
        pixel[1] = j % 255;
        pixel[2] = 0;     
        colorimg.at<cv::Vec3b>(i, j) = pixel;
      }
    }

    cv::namedWindow("grayimg", cv::WINDOW_AUTOSIZE);
    cv::imshow("grayimg", grayimg);
    cv::namedWindow("colorimg", cv::WINDOW_AUTOSIZE);
    cv::imshow("colorimg", colorimg);
    cv::waitKey(0);
    
    return 0;
  }
\end{cppcode}
同样的我们也可以使用指针形式的访问,%
注意因为Mat中的数据并不一定是连续存储的,%
所以我们只能获取每行的指针。%
如:
\begin{cppcode}
  #include <cmath>
  #include <iostream>
  #include "opencv2/opencv.hpp"
  
  int main(int argc, char *argv[]){
    cv::Mat grayimg(600, 800, CV_8UC1);
    cv::Mat colorimg(600, 800, CV_8UC3);
    
    for (int i = 0; i != grayimg.rows; ++i){
      /*! 获取第i行的首地址 */
      uchar *p = grayimg.ptr<uchar>(i);
      for (int j = 0; j != grayimg.cols; ++j){
        p[j] = (i+j)%255;
      }
    }

    for(int i = 0; i != colorimg.rows; ++i){
      /*! 获取第i行的首地址 */
      cv::Vec3b *p = colorimg.ptr<cv::Vec3b>(i);
      for (int j = 0; j != colorimg.cols; ++j){
        p[j][0] = i % 255; /*!< Blue */
        p[j][1] = j % 255; /*!< Green */
        p[j][2] = std::abs(i-j) % 255; /*!< Red */
      }
    }
    
    cv::imshow("grayimg", grayimg);
    cv::imshow("colorimg", colorimg);
    
    cv::waitKey(0);
    
    return 0;
  } 
\end{cppcode}

在OpenCV中同时提供了类似于C++标准库中的迭代器,%
我们同样的使用的这种方式进行访问元素:
\begin{cppcode}
  #include <iostream>
  #include "opencv2/opencv.hpp"

  int main(int argc, char *argv[]){
    cv::Mat grayimg(600, 800, CV_8UC1);
    cv::Mat colorimg(600, 800, CV_8UC3);

    /*! loop gray image */
    for (cv::MatIterator_<uchar> iter = grayimg.begin<uchar>();
    iter != grayimg.end<uchar>();
    ++iter){
      *iter = rand() % 255;
    }

    /*! loop color image */
    for (cv::MatIterator_<cv::Vec3b> iter = colorimg.begin<cv::Vec3b>();
    iter != colorimg.end<cv::Vec3b>();
    ++iter){
      (*iter)[0] = rand() % 123;
      (*iter)[1] = rand() % 255;
      (*iter)[2] = rand() % 255;
    }

    cv::imshow("grayimg", grayimg);
    cv::imshow("colorimg", colorimg);

    cv::waitKey(0);
    return 0;
  }
\end{cppcode}

%%% Local Variables:
%%% mode: latex
%%% TeX-master: t
%%% End:
