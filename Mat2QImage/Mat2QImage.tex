\chapter{Mat和QImage之间的转换}
\section{QImage类}
在Qt包含的类中,其中\textbf{\color{magenta}QImage}类可能是最和计算机视觉最有关的类了,%
QImage提供了像素访问和图像操作的众多函数,%
我们可以很容易的完成QImage对象和OpenCV中Mat对象之间的转换。%

\subsection{QImage的构造函数}
QImage类包含了众多的构造函数允许我们创建QImage对象,%
例如从文件中,从数据中,或者简单先创建一个空的QImage对象,%
然后我们在后面再对每个像素进行操作赋值。%
我们可以创建一个空QImage对象通过给定图像的大小和图片的格式,%
\begin{cppcode}
  QImage image(320, 240, QIMage::Format_RGB888);
\end{cppcode}
这条语句创建了一个空的RGB图像,%
其中拥有320$\times$240个像素。

\subsection{QImage从Mat中创建}
QImage类提供了可以直接从Mat对象创建图片的构造函数,%
但是在使用该构造函数之前我们必须对图像进行一丁点的转换,%
这是因为对于Mat中的图像,%
其格式是采用的BGR,而不是我们通常采用的RGB,%
因此要进行格式的转换。%
如下面的例子:
\begin{cppcode}
  cv::Mat mat = cv::imread("/home/xiaohai/Pictures/test.jpeg");
  cv::cvtColor(mat, mat, CV_BGR2RGB);
  QImage image(mat.data, mat.cols, mat.rows, QImage::Format_RGB888);
\end{cppcode}
其中cv::cvtColor()函数可以完成图像彩色空间的转换。

然而上面的方式并不是推荐的方式,%
我们可以采用下面的构造函数去创建。%
该构造函数需要一个\textbf{\color{magenta}bytesPerLine}参数,%
而该参数的值就是Mat对象中的\textbf{\color{magenta}step}值,%
完整的例子如下:
\begin{cppcode}
  cv::Mat mat = cv::imread("/home/xiaohai/Pictures/test.jpeg");
  cv::cvtColor(mat, mat, CV_BGR2RGB);
  QImage image(mat.data, mat.cols, mat.rows, mat.step, QImage::Format_RGB888);
\end{cppcode}

\subsection{QImage对象转Mat对象}
QImage类包含大量的成员函数,%
这些大量的成员函数可以完成图像的操作和处理,%
如将QImage对象转换称Mat对象,
\begin{cppcode}
  QImage image("/home/xiaohai/Pictures/test.jpeg");
  image = image.convertToFormat(QImage::Format_RGB888);
  Mat mat = Mat(image.height(), image.width(), CV_8UC3, image.bits(), image.bytesPerLine());
\end{cppcode}



%%% Local Variables:
%%% mode: latex
%%% TeX-master: t
%%% End:
